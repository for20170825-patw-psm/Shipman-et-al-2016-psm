\section{BACKGROUND ON PROPENSITY SCORE MATCHING}

\subsection*{Endogeneity, Functional Form Misspecification, and Propensity Score Matching}

\begin{itemize}
 \item 大学の学位($D_i$)が個人の所得($W_i$)に与える影響を検討する。
 \item 実験ならば、大学の学位がある($D_i=1$)と、大学の学位がない($D_i=0$)を無作為に割り当て、グループ間で$W_i$を比較し、平均処置効果(average treatment effect: ATE)を得ることになる。
 \item この場合、所得の決定要因(たとえば、能力や動機)は、大学の学位を得ることと独立であることが仮定され、内生性の懸念はないことになる。
  \begin{itemize}
   \item つまり、$E[W_{0i} | D_i=1]=E[W_{0i} | D_{i}=0]$
  \end{itemize}
 \item ここで、観察的な研究の場合を考え、(1)式を提示する。
\end{itemize}

\begin{equation}
W_i = \beta_0 + \beta_1 D_i + \varepsilon_i
\end{equation}

\begin{itemize}
 \item ここで、所得は、大学在籍の決定要因(たとえば、家庭の事情、能力、動機、キャリアに対する興味。これを$X_i$)からも影響されることが考えられる。
 \item これらの要因をコントロールしないと、$X_i$の影響が(1)式の誤差項($\varepsilon_i$)に入り込んでしまい、$\hat{\beta_1}$は、バイアスのある推定値となる。
  \begin{itemize}
   \item つまり、$E[W_{0i} | D_1=1] \neq E[W_{0i} | D_i=0]$
  \end{itemize}
 \item ここで、(1)式の独立変数に$X_i$を含める
\end{itemize}

\begin{equation}
W_i = \beta_0 + \beta_1 D_i + \beta X_i + \varepsilon_i
\end{equation}

\begin{itemize}
 \item たとえば、知能($IQ_i$)が大学在籍と所得の両方に関連するただひとつの要因だとする。
 \item $IQ_i$を(2)式に含めることで、$\hat{\beta_1}$は、バイアスのない推定値となることが想定される。
  \begin{itemize}
   \item つまり、$X_i$が交絡因子(confounds)を十分に捉えているのなら、(2)式は、バイアスのない推定を行う。
  \end{itemize}
 \item ここで考慮しないといけないのは、$W_i$と$X_i$の関係である。
 \item この関係が不正確(misspecified)に考えられているのなら、“zero conditional mean assumption”($E[\varepsilon | X_i] = 0$)が守られておらず、(2)式は、バイアスのある推定を行うことになる。
  \begin{itemize}
   \item これは、``functional form missepecification(FFM)""と呼ばれる内生性の一形態の問題
  \end{itemize}
\item マッチングは、FFMの問題を緩和するのに有効である。
 \item つまり、大学の学位を有する人($D_i=1$)を、同程度の$IQ_i$($X_i$)である大学の学位を有していない人($D_i=0$)をマッチングし、処置群と対照群の差異を減少させる。
  \begin{itemize}
   \item こうすれば、$W_i$と$X_i$の関係の仮定を置く必要なく、$IQ_i$($X_i$)の影響を調整できる。
  \end{itemize}
 \item 会計研究のようなアーカイバルデータを用いた研究では、処置群と対照群への割り当ての決定要因は複数存在する。
 \item Rosendaum and Rubin(1983)は、$X_i$が複数想定される場合に、処置群に割り当てられる確率を$X_i$にもとづいて求め、その確率を用いてマッチングを行う手法を提案した。
  \begin{itemize}
   \item その確率のことを``傾向スコア(propensity score)''という。
  \end{itemize}
\item 傾向スコアは、以下の(3)式で推定される。
\end{itemize}

\begin{equation}
D_i = \beta_0 + \beta_1 X_i + \varepsilon_i
\end{equation}

\begin{itemize}
 \item 処置群($D_i=1$)に属する観測値は、(3)式で推定された傾向スコアが近似している対照群($D_i=0$)に属する観測値とマッチングされる。
 \item つまり、処置群と対照群は、$X_i$が類似している観測値同士の組み合わせとなる。
  \begin{itemize}
   \item これは、$D_i$と$X_i$の相関を最小にし、FFMの懸念を緩和する。
  \end{itemize}
\end{itemize}

\paragraph{マッチングの質と外的妥当性}

\begin{itemize}
 \item PSMは、“common supprt”(あるいはoverlap)の範囲内に存在する観測値をサンプルとすることになる。
 \item もし、$IQ_i$が大学の学位取得に強く影響を与えるのなら、マッチングは、大学の学位を有して(有してなくて)、かつ$IQ_i$の高い(低い)観測値の多くを排除する可能性がある。
  \begin{itemize}
   \item つまり、$IQ_i$と$D_i$間の関連性が強まるごとに、マッチングの質の程度が減少する。
  \end{itemize}
 \item PSMの外的妥当性は、サンプルのATEと母集団のATEの近似性で判断されるが、サンプルサイズが小さくなることは、その妥当性に影響を与えるかもしれない(Cram, Karan, and Stuart 2009; Heckman, Ichimura, and Todd 1998)。
\end{itemize}
 
\paragraph{以上の例の重要な仮定}

\begin{enumerate}
 \item モデルには、大学の在籍と所得のいずれにも関連する、すべての要因が含まれている。
 \item 共変量(covariates)は、以上の要因から正確に計算できる。
\end{enumerate}
\begin{itemize}
 \item しかし、実証分析では、要因の特定や測定が困難であることから、以上の仮定が必ず遵守されるわけではない。
 \item このことは、MRとPSMのいずれにもバイアスがもたらされていることを意味する。
  \begin{itemize}
   \item 以下でこの限界について議論を行う。
  \end{itemize}
\end{itemize}

\subsection*{Propensity Score Matching in Accounting ResearchーMisconceptions and Limitations}

\paragraph{MRに対するPSMの優位性}

\begin{itemize}
 \item MRは、ATEを推定するのに柔軟なフレームワークを提供しており、観察的な研究のスタートポイントとなる。
 \item しかし近年、MRの頑健性を証明するためや、FFMの問題を緩和するために、PSMを用いる研究が存在している(Armstrong, Ittner, and Larcker 2012b; Armstrong, Jagolinzer, and Larcker 2010; Lawrence et al. 2011; Miunutti-Meza 2013)。
 \item 例を挙げると、クライアント企業の規模とBig 4の監査法人を選択することには、強い関連性があると考えられる。
 \item この場合、MRのモデルにクライアント企業の規模を組み込みコントロールしようとする。
  \begin{itemize}
   \item しかし、そのコントロール変数と結果変数との関連の特定が不適切である場合、Big 4の監査法人を選択した効果についての推定にバイアスがもたらされる。
  \end{itemize}
 \item PSMは、企業規模や他の要因を用いてマッチングを行うことで、クライアント企業の規模と監査法人の選択の効果を最小にする。
  \begin{itemize}
   \item この場合、変数同士の関係の仮定を置く必要がなく、PSMはFFMの問題を緩和することになる。
  \end{itemize}
\end{itemize}

\paragraph{先行研究の誤解}

\begin{itemize}
 \item 以上の優位さを認識している先行研究は数少ない。
 \item 先行研究は、``内生性(endogeneity)''``(自己)選択バイアス((self)selection bias)''``欠落変数バイアス(omitted variable bias)''を広く是正する手法として、あるいは、``操作変数法(instrumental variables approaches)''の代替法として、PSMを用いている。
 \item しかし、MRと同様に、PSMは、自己選択や内生性の懸念に対処するような、処置と結果に関連するすべての要素を正確に特定したり測定したりする能力はない。
  \begin{itemize}
   \item つまり、PSMがHeckman(1979)のタイプの選択モデルの代用ではなく、内生性、欠落変数、あるいは自己選択に関連する広い懸念を緩和するわけではない。
  \end{itemize}
\end{itemize}

\paragraph{PSMは、実験研究を完全に再現するわけではない}

\begin{enumerate}
 \item 観察不能な要因の存在は、観察可能な要因のみでマッチングした場合に、処置への割り当てが完全にランダムになるわけではないことを示している。
  \begin{itemize}
   \item 実験研究は、処置への割り当てが完全にランダムなので、観察可能な要因「と」観察不能な要因の両方ともコントロールできる。
  \end{itemize}
 \item 実験とは異なり、PSMは、観測値を処置に実際に割り当てているわけではない。
  \begin{itemize}
   \item PSMは、あくまでサンプルの選択や重み付けを行なっている。
  \end{itemize}
\end{enumerate}

\paragraph{マッチングによってサンプルサイズが小さくなる問題}
\begin{itemize}
 \item マッチングは、処置群と対照群が重なり合う(overlapping)する部分で主として生じ、その範囲外のサンプルはマッチングされない可能性が高い。
 \item その範囲においてさえも、リサーチ・デザインの設定によってマッチングされないサンプルが生じることもある。
 \begin{itemize}
  \item マッチングでサンプルサイズが小さくなることは、外的妥当性についての問題を有することになる。
 \end{itemize}
 \item 4節の分析のサンプルを見ると、内部統制の弱さ(internal control weaknesses: ICW)を監査報告書で指摘されているのは、サンプルのわずか7\%にすぎない。
 \item この状況において、一般的な会計研究のように1対1でマッチングを行うと、内部統制の弱さを指摘されていない観測値をサンプルから除外することになる。
 \item 除外された観測値には、マッチングされても何らおかしくない観測値も含まれているはずである。
 \item 実際、内部統制の弱さについては、overlappingしている部分が大きいにもかかわらず、内部統制の弱さの指摘のない観測値のうち、サンプルとして選択されるのは7.5\%しかない。
 \item この場合、マッチングごとに選択されるサンプルが変わってしまう可能性があるので、再現性に問題が生じてしまう。
 \item 研究ごとに同様の結果が確認されることが重要となってくる(DeFond et al.[2015]が指摘している)。
\end{itemize}

\paragraph{PSMを手法として用いる時の注意}

\begin{itemize}
 \item 以上の問題は、マッチングされたサンプルで分析する妥当性、PSMとMRの結果の相違時の解釈、および母集団のATEとサンプルのATEを比較することで判断される結果の一般化に影響を与える。
 \item 観察的な研究では、分析手法を事後的に(post hoc)選択することが可能なので、PSMを軽々に使ってしまいがちになる。
  \begin{itemize}
   \item しかし、基本的には先行研究のリサーチデザインにならい、PSMを用いる場合は、その妥当性を判断した上で利用するべきである。
  \end{itemize}
\end{itemize}

\subsection*{Primary Design Choices in Propensity Score Matching}

\begin{itemize}
 \item PSMの手法は標準化されてないため、同じデータを用いたとしても、研究ごとにその結果が異なる可能性がある(Angrist and Pischke, 2009, 86)。
 \item 以下で、PSMの手法について議論する。
\end{itemize}

\subsubsection*{Primary Design Choices for Estimating the Propensity Score}

\begin{enumerate}
 \item 処置群と対照群の区別
   \begin{itemize}
    \item 観測値を処置群と対照群に割り当てる場合、その割り当ての基準に問題が生じる。
    \item たとえば、IFRSを適用している企業とそうでない企業のような、はっきりと2つに分けられるものについては、割り当ての際に問題が生じないだろう。
    \item 一方、監査法人の規模、アナリストのカバレッジ、あるいは経営者報酬のような場合、処置群と対照群の分かれ目であるカットオフの時点の選択の際に問題が生じる。
    \item また、このような連続的な要因で割り当てを行う場合、マッチングは、カットオフの付近にある観測値同士で行われる可能性が高まる。
     \begin{itemize}
     \item この場合、第二種の過誤の可能性が高まってしまう。
    \end{itemize}
   \end{itemize}
 \item 交絡因子の特定
   \begin{itemize}
    \item 交絡因子(confounding factors)の選択は、サンプルの構成や結果に影響を与える。
    \item 交絡因子は、理論的な裏付けによって選択されるべきである。
    \item モデルの適合度(fit)や予測力(predicive power)にもとづくべきというのは誤解である(Peel and Makepeace,2012)。
    \item また、PSMは、MRと同様の変数を用いるべきであり、当該変数が理論によってMRに導入するべきでないとするなら、PSMにも導入するべきではない。
    \item MRとPSMで変数が異なることは、内的な一貫性がないことや事後的なリサーチデザインの選択という点から批判を招く恐れがある。
   \end{itemize}
\end{enumerate}

\subsubsection*{Primary Design Choices for Forming the Matched Sample}

\begin{enumerate}
 \item マッチングの置き換えを認めるか認めないか
   \begin{itemize}
    \item マッチングの置き換えを認めないとは、マッチングを1回のみ行うことである。
    \item この場合、ある対照群の観測値が、処置群の観測値いくつかとマッチングすることが最良としても、そのうちのひとつの処置群の観測値としかマッチングしないことになってしまう。
    \begin{itemize}
     \item そうなると、マッチングの質が低くなったり、置換を認めている場合よりサンプルサイズが小さくなったりする。
    \end{itemize}
    \item マッチングの置き換えを認める場合、ある対照群の観測値は、複数の処置群の観測値とマッチングされる。
    \item 1回だけマッチングを行う場合よりも、バイアスが低減されたり、サンプルサイズが大きくなったりする可能性がある。
%以下、実際の分析で置換マッチングをしている場面がありますので、そこを読んでからまとめます。
    \item 置き換えを認めたマッチングの際、生じたマッチングの数に応じて、ATEや標準誤差に重み付けや調整を行う必要がある(Armstrong et al. 2010; Stuart 2010)。
   \end{itemize}
 \item キャリパー距離(Caliper Distance)
   \begin{itemize}
    \item 処置群と対照群がマッチングする傾向スコアの距離を制限する。
     \begin{itemize}
      \item 傾向スコアの距離が過度に大きいのにマッチングされてしまい、マッチングの質が低下する問題を防ぐことができる。
    \end{itemize}
   \end{itemize}
 \item 「1対1」か「1対多」か
   \begin{itemize}
    \item 会計研究では、処置群と対照群の観測値を1対1でマッチングするのが主流である。
    \item しかし、“common support”の部分に対照群の観測値が処置群の観測値よりも多く存在する場合、1対多のマッチングの方が有効となるかもしれない。
     \begin{itemize}    
      \item この部分に、処置群の反事実(counterfactual)となりうる対照群の観測値が多く存在すると考えられるから
     \end{itemize}
    \item 「1対多」のマッチングは、マッチングの質の低下を招く恐れがある。
    \item しかし、本稿とDeFond et al.(2015)で指摘するように、サンプルの変動という問題を部分的に緩和するかもしれない。
    \item また、置換を認めたマッチングと同様に、1対多のマッチングを行なった場合、観測値を重み付けするべきである。
   \end{itemize}
\end{enumerate}

\subsubsection*{Evaluating Matched Sample}
\begin{enumerate}
 \item マッチングの質の評価
  \begin{itemize}
   \item PSMは、処置群と対照群の共変量を``バランスさせる(balancing)''。
   \item しかし、PSMは、常に完璧なマッチングを行うとは限らない(特に、割り当てが連続変数による場合)。
    \begin{itemize}
     \item したがって、マッチングに対する評価を行う必要がある。
    \end{itemize}
   \item 一般的には、処置群と対照群の共変量の平均値や中央値を検定することで、バランスしているかどうかを評価する。
   \item しかし、この差異が非有意であろうとも、FFMによるバイアスを除去できているとは限らない。
   \item 一方、この差異が有意であったとしても、マッチングしていない時よりは差異は小さいだろうし、FFMのバイアスも緩和されている可能性はある。
    \begin{itemize}
     \item 共変量のバランスは、処置群と対照群の共変量の差の大きさ、``および''検定によって確認される差のインパクトによって評価されるべきである。
    \end{itemize}
  \end{itemize}
\end{enumerate}
 
\subsubsection*{Estimating the Treatment Effect}
\begin{enumerate}
 \item $t$検定かMRか
   \begin{itemize}
    \item マッチングののち、ATEは、単純に$t$検定、もしくはMRにより評価されることになる。
    \item 共変量のバランスが達成されているのなら、$t$検定で妥当かもしれない。
    \item そうでないのなら、処置群と対照群に残っている共変量の相違をコントロールするために、MRを用いることが推奨される(Ho et al. 2007; Lawrence et al. 2011)。
   \end{itemize}
\end{enumerate}
