\section{Introduction}

\paragraph{研究の背景: 内生性の問題に対する従来の手法の問題点}
\begin{itemize}
 \item 実証的会計研究において、因果処置効果 (causal treatment effects) を推定することが、
       主たる目的とされることがある (Gow, Larcker, and Reiss, 2016)。
 \item 非実験的データを利用した研究では、
       処置群が無作為割り当てではない (non-random treatment assignment) ために、
       内生性の問題が生じてしまう。
 \item 内生性を緩和させるための従来の手法
       \begin{itemize}
        \item アーカイバル研究では、伝統的には
              重回帰 (multiple regression, MR) モデルを利用
        \item ただし、MRを利用してバイアスのかかっていない推定値を得るためには、
              結果変数 (Y) と説明変数 (X) の関係に適切な仕様 (前提) を満たす必要がある。
        \item YとXの関係が前提を満たしていない場合、
              MRは``回帰式の特定ミス (functional form misspecification, FFM)''の影響を受け、
              バイアスのかかった推定値を得ることになり得る。
        \item FFMによる潜在的なバイアスは、処置群 (treatment groups) が似ていないほど、
              強くなってしまう。
       \end{itemize}
\end{itemize}

\paragraph{Propensity score matching (PSM)}
\begin{itemize}
 \item 変数間の関係の特定に対する依存 (reliance) を減少させることで、
       上記のような問題を緩和
 \item 処置群から推定された傾向 (likelihood; 尤度?) を用いて、様々な側面
       から、処置群に属する観測値と対照群をマッチング
 \item 変数間の関係 (functional relation) に関する
       緩い (relaxed) 仮定のもとで、処置効果を直接的かつ直観的に推定可能
 \item ただし、
       PSMは伝統的なMRの手法と比較して、理論的有用性 (theoretical
       benefits) が少ない
\end{itemize}

\paragraph{本論文の検討事項}
\begin{itemize}
 \item 内生性、MRの手法、FFMに関する問題、マッチング手法のメリット、PSMの設定のインプリケーションについて議論
 \item 以下の雑誌に掲載されている、PSMを利用した86本の論文に対する議論
       \begin{itemize}
        \item \textit{The Accounting Review},\textit{Contemporary Accounting Research},\textit{Journal of Accounting and Economics},
              \textit{Journal of Accounitng Reserach},\textit{Review of Accounting Studies}
        \item PSMを利用した論文数は増加 (2008年: 0本、2014年: 26本)
        \item 各論文について、PSMの利用および設定が正当なものであるのかを評価
       \end{itemize}
 \item 財務報告に関する研究における3つの設定において、PSMを利用した場合に問題が生じる例
       \begin{enumerate}
        \item 監査人の規模
        \item 内部統制の弱さ
        \item フォローしているアナリスト数
       \end{enumerate}
 \item マッチング手法の利用もしくは、FFMに基づく内生性問題に取り組む他の手法の利用を検討している研究に対するいくつかの提言
\end{itemize}

\paragraph{本論文の貢献}
\begin{itemize}
 \item DeFond, Erkens and Zhang (2015) を補完
       \begin{itemize}
        \item DeFond, Erkens and Zhang (2015)
              \begin{itemize}
               \item マッチング手法を利用して、監査人の規模および監査の質の関係について分析
               \item 設定 (design) をランダムに数千回実行し、総合的な結果を提示
               \item 分析結果の多くは、4大監査法人は、それ以外の監査法人よりも優れた監査の質を提供していることを示唆
               \item Lawrence, Minutti-Meza and Zhang (2011) の結果と整合
              \end{itemize}
        \item 本論文とDeFond et al. (2015) の違い
              \begin{itemize}
               \item DeFond et al. (2015) の主たる目的は、4大監査法人の影響に関する実証的証拠を提供すること
               \item 本研究も同様のテーマを共有
               \item ただし、
                     会計研究においてPSMを利用している論文をレビューし、
                     PSMの有用性および限界に関する議論を提供し、
                     一般的な会計研究における設定 (setting) の、
                     固有の設定 (design choices) に関する影響について明示する、という点で異なる。
               \item 本論文におけるどの設定 (settings) においても、実証的証拠を提供するものではない。
              \end{itemize}
       \end{itemize}
 \item PSMの利用を検討している将来の研究に対する情報提供
\end{itemize}

\paragraph{構成}
\begin{description}
 \item[Introduction] \mbox{}\\
            本論文の目的と意義 (担当: 久多里)
 \item[Background on propensity score matching] \mbox{}\\
            PSMの有用性、誤解、適切なリサーチ・デザインの設計に関する議論 (担当: 井上)
 \item[Propensity score matching in accounting research] \mbox{}\\
            主要な会計雑誌に掲載されたPSMを利用した研究のサーベイ (担当: 大洲)
 \item[Empirical examples of propensity score matching in accounitng settings] \mbox{}\\
            3つの会計的な設定を例として、PSMが引き起こす問題に関する実例 (担当: 井上)
 \item[Suggestion and consideration for future research] \mbox{}\\
            マッチング手法の利用を検討している研究に対する提言  (担当: 久多里)
 \item[Conclusion] \mbox{}\\
            本論文の発見事項とインプリケーション  (担当: 大洲)
\end{description}


% \paragraph{問題意識: 内生性の問題に対する従来の手法}
% \begin{itemize}
%  \item 実証的会計研究において、因果的処置効果 (causal treatment effects) を推定することが、
%        主たる目的とされることがある (Gow, Larcker, and Reiss, 2016)。
%  \item 非実験的データ (観測されたデータ?) を利用した研究は、
%        処置群が無作為割り当てではないこと (non-random treatment assignment、マッチングサンプルがランダムにわりあてられていない、
%        ということを言いたい?) によって、内生性の問題を生じさせてしまう。
%  \item 伝統的には、観測されたデータにおける内生性を緩和させるために、アーカイバル研究では
%        重回帰 (multiple regression, MR) モデルを利用している。
%  \item しかしながら、MRを利用してバイアスのかかっていない推定値を得るためには、
%        結果変数 (Y) と説明変数 (X) の関係に適切な仕様 (誤差の分散が1で平均が0とか…) が認められる必要がある。
%  \item YとXの関係が誤って指定されたものである場合、
%        MRは``functional form misspecification (FFM)''の影響を受けることになり、
%        バイアスのかかった推定値を得ることになり得る。
%  \item FFMによる潜在的なバイアスは、処置群 (treatment groups) が似ていないほど、
%        強くなってしまう。
% \end{itemize}


% \paragraph{問題意識}

% \begin{itemize}
%  \item Propensity score matching (PSM)
%        \begin{itemize}
%         \item 変数間の特定の関係に対する依存を減少させることで、
%               上記のような問題を緩和
%         \item 機械的に、PSMは、処置群から選択された観測値と、
%               処置を受ける傾向 (likelihood) の推定値を利用して様々な側面から判断したコントロール・グループを
%               マッチさせる。
%         \item PSMの反事実的本質により、変数間の関数関係に関する
%               緩い (relaxed) 仮定のもとで、処置効果を直接的かつ直感的に推定することができる。
%         \item しかしながら、FFMの影響を緩和する以外に、PSMは伝統的なMRの手法が持つ理論的有用性を損なうことになる。
%        \end{itemize}
% \end{itemize}

% \paragraph{本論文の検討事項}
% \begin{itemize}
%  \item はじめに、内生性、MRの手法、FFMに関する問題、マッチング手法のメリットについて議論
%  \item いくつかのPSMのリサーチ・デザインの設計のインプリケーションについて議論
%  \item 以下の雑誌に掲載されているPSMを利用している86本の論文について
%        \begin{itemize}
%         \item \textit{The Accounting Review},\textit{Contemporary Accounting Research},\textit{Journal of Accounting and Economics},
%               \textit{Journal of Accounitng Reserach},\textit{Review of Accounting Studies}
%        \end{itemize}
%  \item PSMを利用した論文は、2008年には0本だったのに対し、2014年には26本とかなり増加しており、
%        他の手法を上回って、PSMに対するアクセプトが増加している (あるいは、同じくらい好まれている) ことを示唆
%  \item 実際に、今回サーベイした論文のうち22本は、少なくとも1つの仮説を検証するために、
%        PSMのみを分析手法として採用している。
%  \item 新たに採用されたどの手法について言えることだが、
%        PSMのメットと限界を適切に理解し、正確に適用することが重要である。
%  \item 各論文について、PSMの利用および分析手法 (design choiced) が正当なものであるのかを評価する。
%  \item 懸念事項として、
%        20本の論文しか、PSMを利用する動機として、FFMを特定 (adress) するため、あるいは、MRの分析における
%        線形仮定を緩和させるためであると明記していない。
%  \item しかしながら、33本の論文は、PSMを利用することの動機として、
%        内生性に関する一般的問題 (concerns) や、自己選択問題 (self-selection)、欠落変数バイアスについて触れている。
%  \item 分析手法は会計研究において一般的でない、もしくは、
%        明記されていることはほとんどないことも確認された。
%  \item 総合的にみて、会計研究では、研究の特性に対する全体的な評価を行なうことなく、PSMが実施されることが多い。
%        そのような誤解が、PSMの利用が飛躍的に増加したことを部分的に説明している (皆ちゃんと
%        理解せずに使ってるから、すごくPSMを利用した研究が増えたんだよねーたぶん)。
% \end{itemize}

% \paragraph{財務報告における3つのsettingsにおけるPSMを利用した事例}
% \begin{enumerate}
%  \item 監査人の規模
%  \item 内部統制の弱さ
%  \item フォローしているアナリスト数
% \end{enumerate}

% \begin{itemize}
%  \item 連続した処置構成を二分するような、ある一般的なdesign choiceは、
%        コントロールグループの処置水準が処置群の処置水準と似かよっている
%        マッチサンプルを生みだす傾向があり、そのため、効果量を減少させる。
%  \item PSMの分析においてreduced sample size inherentに加えて、
%        検定力を大幅に減少させ、偽陰性 (False negative: 対立仮説が真なのに帰無仮説を採択してしまうこと) の
%        確率を高めてしまう。
%  \item PSMのdesignにおいて、
%        問題ないように思われる変更が、サンプルの構成および推定に対して、どのように
%        多大な影響を与えるのかを、デモンストレーションする。
%  \item 他の手法を利用した場合の研究は、それ自体は、正当性があるものの、
%        固有の性質をマッチングすることは、処置効果の程度もしくは存在について、
%        異なる結論に帰着させるかもしれない。
%  \item つまり、PSMを利用した研究は、発見事項が今回利用したリサーチ・デザインに依存するものではなく、
%        処置効果を推定する他の方法を利用した場合にも頑健であることを厳密に証明するべきである (Leamer 1983)。
%  \item 本論文の結論は、マッチング手法の利用もしくはFFMに基づく内生性問題に取り組む他の手法の利用を考慮する研究に対するいくつかの提言
%        を提供する。
% \end{itemize}

% \begin{itemize}
%  \item 本論文は、マッチング手法を利用した監査人の規模および監査の質の関係について分析したDeFond, Erkens and Zhang (2015) を
%        補完するものである。
%        \begin{itemize}
%         \item designをランダムに数千回実行した総合的な結果を提示
%         \item 分析結果の多くは、4大監査法人は、それ以外の監査法人よりも優れた監査の質を提供していることを示唆
%         \item Lawrence, Minutti-Meza and Zhang (2011) の結果と整合
%         \item DeFond et al. (2015) の主な目的は、4大監査法人の影響に関する実証的証拠を提供すること
%        \end{itemize}
%  \item 本研究も同様のテーマを共有しているが、
%        会計研究においてPSMを利用している論文をレビューし、
%        PSMの有用性および限界に関する議論を提供し、
%        一般的なaccounting settingsにおける固有のdesign choicesの影響について明示する、という点で異なる
%  \item DeFond et al. (2015) とは違い、
%        本論文におけるどのsettingsにおいても、実証的証拠を提供するものではない。
%  \item さらに、本論文の主眼はPSMの利用を検討している将来の研究に対して情報を提供することにある。
% \end{itemize}

% \paragraph{構成}
% \begin{enumerate}
%  \item Background on propensity score matching: PSMの有用性、誤解、適切なリサーチ・デザインの設計に関する議論
%  \item Propensity score matching in accounting research: 主な会計雑誌に掲載されたPSMを利用した研究のサーベイ
%  \item Empirical examples of propensity score matching in accounitng settings: 3つのリサーチ・を設定した場合における、PSMが引き起こす問題に関するデモンストレーション
%  \item Suggestion and consideration for future research: マッチング手法に関する今後の課題の提示 
%  \item Conclusion: 本論文の発見事項とインプリケーション 
% \end{enumerate}egin{itemize}
%         \item アーカイバル研究では、伝統的には
%               重回帰 (multiple regression, MR) モデルを利用
%         \item ただし、MRを利用してバイアスのかかっていない推定値を得るためには、
%               結果変数 (Y) と説明変数 (X) の関係に適切な仕様 (前提) を満たす必要がある。
%         \item YとXの関係が前提を満たしていない場合、
%               MRは``回帰式の特定ミス (functional form misspecification, FFM)''の影響を受けることになり、
%               バイアスのかかった推定値を得ることになり得る。
%         \item FFMによる潜在的なバイアスは、処置群 (treatment groups) が似ていないほど、
%               強くなってしまう。
%        \end{itemize}
% \end{itemize}

% \paragraph{Propensity score matching (PSM)}
% \begin{itemize}
%  \item 変数間の特定の仮定を少なくすることで、
%        上記のような問題を緩和
%  \item 処置群から選択された観測値と
%        処置を受ける傾向の推定値を利用して、機械的に、様々な側面から判断したコントロール・グループを
%        マッチ
%  \item 変数間の関係 (functional relation) に関する
%        緩い (relaxed) 仮定のもとで、処置効果を直接的かつ直感的に推定可能
%  \item ただし、
%        PSMは伝統的なMRの手法が持つ理論的有用性を損なう。
% \end{itemize}

% \paragraph{本論文の検討事項}
% \begin{itemize}
%  \item 内生性、MRの手法、FFMに関する問題、マッチング手法のメリット、PSMの設定のインプリケーションについて議論
%  \item 以下の雑誌に掲載されている、PSMを利用した86本の論文に対する議論
%        \begin{itemize}
%         \item \textit{The Accounting Review},\textit{Contemporary Accounting Research},\textit{Journal of Accounting and Economics},
%               \textit{Journal of Accounitng Reserach},\textit{Review of Accounting Studies}
%         \item PSMを利用した論文数は増加 (2008年: 0本、2014年: 26本)
%         \item 各論文について、PSMの利用および設定が正当なものであるのかを評価
%        \end{itemize}
%  \item 財務報告に関する研究における3つの設定において、PSMを利用した場合に問題が生じる例
%        \begin{enumerate}
%         \item 監査人の規模
%         \item 内部統制の弱さ
%         \item フォローしているアナリスト数
%        \end{enumerate}
%  \item マッチング手法の利用もしくは、FFMに基づく内生性問題に取り組む他の手法の利用を検討している研究に対するいくつかの提言
% \end{itemize}

% \paragraph{本論文の貢献}
% \begin{itemize}
%  \item DeFond, Erkens and Zhang (2015) を補完
%        \begin{itemize}
%         \item DeFond, Erkens and Zhang (2015)
%               \begin{itemize}
%                \item マッチング手法を利用して、監査人の規模および監査の質の関係について分析
%                \item 設定 (design) をランダムに数千回実行し、総合的な結果を提示
%                \item 分析結果の多くは、4大監査法人は、それ以外の監査法人よりも優れた監査の質を提供していることを示唆
%                \item Lawrence, Minutti-Meza and Zhang (2011) の結果と整合
%               \end{itemize}
%         \item 本論文とDeFond et al. (2015) の違い
%               \begin{itemize}
%                \item DeFond et al. (2015) の主たる目的は、4大監査法人の影響に関する実証的証拠を提供すること
%                \item 本研究も同様のテーマを共有
%                \item ただし、
%                      会計研究においてPSMを利用している論文をレビューし、
%                      PSMの有用性および限界に関する議論を提供し、
%                      一般的な会計研究における設定 (setting) の、
%                      固有の設定 (design choices) に関する影響について明示する、という点で異なる。
%                \item 本論文におけるどの設定 (settings) においても、実証的証拠を提供するものではない。
%               \end{itemize}
%        \end{itemize}
%  \item PSMの利用を検討している将来の研究に対する情報提供
% \end{itemize}

% \paragraph{構成}
% \begin{enumerate}
%  \item Background on propensity score matching: PSMの有用性、誤解、適切なリサーチ・デザインの設計に関する議論
%  \item Propensity score matching in accounting research: 主な会計雑誌に掲載されたPSMを利用した研究のサーベイ
%  \item Empirical examples of propensity score matching in accounitng settings: 3つの会計的な例を設定した場合における、PSMが引き起こす問題に関する実例
%  \item Suggestion and consideration for future research: マッチング手法の利用を検討している研究に対する提言 
%  \item Conclusion: 本論文の発見事項とインプリケーション 
% \end{enumerate}


% \paragraph{問題意識: 内生性の問題に対する従来の手法}
% \begin{itemize}
%  \item 実証的会計研究において、因果的処置効果 (causal treatment effects) を推定することが、
%        主たる目的とされることがある (Gow, Larcker, and Reiss, 2016)。
%  \item 非実験的データ (観測されたデータ?) を利用した研究は、
%        処置群が無作為割り当てではないこと (non-random treatment assignment、マッチングサンプルがランダムにわりあてられていない、
%        ということを言いたい?) によって、内生性の問題を生じさせてしまう。
%  \item 伝統的には、観測されたデータにおける内生性を緩和させるために、アーカイバル研究では
%        重回帰 (multiple regression, MR) モデルを利用している。
%  \item しかしながら、MRを利用してバイアスのかかっていない推定値を得るためには、
%        結果変数 (Y) と説明変数 (X) の関係に適切な仕様 (誤差の分散が1で平均が0とか…) が認められる必要がある。
%  \item YとXの関係が誤って指定されたものである場合、
%        MRは``functional form misspecification (FFM)''の影響を受けることになり、
%        バイアスのかかった推定値を得ることになり得る。
%  \item FFMによる潜在的なバイアスは、処置群 (treatment groups) が似ていないほど、
%        強くなってしまう。
% \end{itemize}


% \paragraph{問題意識}

% \begin{itemize}
%  \item Propensity score matching (PSM)
%        \begin{itemize}
%         \item 変数間の特定の関係に対する依存を減少させることで、
%               上記のような問題を緩和
%         \item 機械的に、PSMは、処置群から選択された観測値と、
%               処置を受ける傾向 (likelihood) の推定値を利用して様々な側面から判断したコントロール・グループを
%               マッチさせる。
%         \item PSMの反事実的本質により、変数間の関数関係に関する
%               緩い (relaxed) 仮定のもとで、処置効果を直接的かつ直感的に推定することができる。
%         \item しかしながら、FFMの影響を緩和する以外に、PSMは伝統的なMRの手法が持つ理論的有用性を損なうことになる。
%        \end{itemize}
% \end{itemize}

% \paragraph{本論文の検討事項}
% \begin{itemize}
%  \item はじめに、内生性、MRの手法、FFMに関する問題、マッチング手法のメリットについて議論
%  \item いくつかのPSMのリサーチ・デザインの設計のインプリケーションについて議論
%  \item 以下の雑誌に掲載されているPSMを利用している86本の論文について
%        \begin{itemize}
%         \item \textit{The Accounting Review},\textit{Contemporary Accounting Research},\textit{Journal of Accounting and Economics},
%               \textit{Journal of Accounitng Reserach},\textit{Review of Accounting Studies}
%        \end{itemize}
%  \item PSMを利用した論文は、2008年には0本だったのに対し、2014年には26本とかなり増加しており、
%        他の手法を上回って、PSMに対するアクセプトが増加している (あるいは、同じくらい好まれている) ことを示唆
%  \item 実際に、今回サーベイした論文のうち22本は、少なくとも1つの仮説を検証するために、
%        PSMのみを分析手法として採用している。
%  \item 新たに採用されたどの手法について言えることだが、
%        PSMのメットと限界を適切に理解し、正確に適用することが重要である。
%  \item 各論文について、PSMの利用および分析手法 (design choiced) が正当なものであるのかを評価する。
%  \item 懸念事項として、
%        20本の論文しか、PSMを利用する動機として、FFMを特定 (adress) するため、あるいは、MRの分析における
%        線形仮定を緩和させるためであると明記していない。
%  \item しかしながら、33本の論文は、PSMを利用することの動機として、
%        内生性に関する一般的問題 (concerns) や、自己選択問題 (self-selection)、欠落変数バイアスについて触れている。
%  \item 分析手法は会計研究において一般的でない、もしくは、
%        明記されていることはほとんどないことも確認された。
%  \item 総合的にみて、会計研究では、研究の特性に対する全体的な評価を行なうことなく、PSMが実施されることが多い。
%        そのような誤解が、PSMの利用が飛躍的に増加したことを部分的に説明している (皆ちゃんと
%        理解せずに使ってるから、すごくPSMを利用した研究が増えたんだよねーたぶん)。
% \end{itemize}

% \paragraph{財務報告における3つのsettingsにおけるPSMを利用した事例}
% \begin{enumerate}
%  \item 監査人の規模
%  \item 内部統制の弱さ
%  \item フォローしているアナリスト数
% \end{enumerate}

% \begin{itemize}
%  \item 連続した処置構成を二分するような、ある一般的なdesign choiceは、
%        コントロールグループの処置水準が処置群の処置水準と似かよっている
%        マッチサンプルを生みだす傾向があり、そのため、効果量を減少させる。
%  \item PSMの分析においてreduced sample size inherentに加えて、
%        検定力を大幅に減少させ、偽陰性 (False negative: 対立仮説が真なのに帰無仮説を採択してしまうこと) の
%        確率を高めてしまう。
%  \item PSMのdesignにおいて、
%        問題ないように思われる変更が、サンプルの構成および推定に対して、どのように
%        多大な影響を与えるのかを、デモンストレーションする。
%  \item 他の手法を利用した場合の研究は、それ自体は、正当性があるものの、
%        固有の性質をマッチングすることは、処置効果の程度もしくは存在について、
%        異なる結論に帰着させるかもしれない。
%  \item つまり、PSMを利用した研究は、発見事項が今回利用したリサーチ・デザインに依存するものではなく、
%        処置効果を推定する他の方法を利用した場合にも頑健であることを厳密に証明するべきである (Leamer 1983)。
%  \item 本論文の結論は、マッチング手法の利用もしくはFFMに基づく内生性問題に取り組む他の手法の利用を考慮する研究に対するいくつかの提言
%        を提供する。
% \end{itemize}

% \begin{itemize}
%  \item 本論文は、マッチング手法を利用した監査人の規模および監査の質の関係について分析したDeFond, Erkens and Zhang (2015) を
%        補完するものである。
%        \begin{itemize}
%         \item designをランダムに数千回実行した総合的な結果を提示
%         \item 分析結果の多くは、4大監査法人は、それ以外の監査法人よりも優れた監査の質を提供していることを示唆
%         \item Lawrence, Minutti-Meza and Zhang (2011) の結果と整合
%         \item DeFond et al. (2015) の主な目的は、4大監査法人の影響に関する実証的証拠を提供すること
%        \end{itemize}
%  \item 本研究も同様のテーマを共有しているが、
%        会計研究においてPSMを利用している論文をレビューし、
%        PSMの有用性および限界に関する議論を提供し、
%        一般的なaccounting settingsにおける固有のdesign choicesの影響について明示する、という点で異なる
%  \item DeFond et al. (2015) とは違い、
%        本論文におけるどのsettingsにおいても、実証的証拠を提供するものではない。
%  \item さらに、本論文の主眼はPSMの利用を検討している将来の研究に対して情報を提供することにある。
% \end{itemize}

% \paragraph{構成}
% \begin{enumerate}
%  \item Background on propensity score matching: PSMの有用性、誤解、適切なリサーチ・デザインの設計に関する議論
%  \item Propensity score matching in accounting research: 主な会計雑誌に掲載されたPSMを利用した研究のサーベイ
%  \item Empirical examples of propensity score matching in accounitng settings: 3つのリサーチ・を設定した場合における、PSMが引き起こす問題に関するデモンストレーション
%  \item Suggestion and consideration for future research: マッチング手法に関する今後の課題の提示 
%  \item Conclusion: 本論文の発見事項とインプリケーション 
% \end{enumerate}
