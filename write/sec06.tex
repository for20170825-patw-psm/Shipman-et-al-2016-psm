\section{CONCLUSION}

\begin{itemize}
 \item Shipman, Swanquist, and Whited の分析対象
       \begin{itemize}
        \item PSM の理論的基礎を議論し、昨今の会計研究における PSM の利
              用を調査し、そして、デザインの選択 (design choice) にかん
              する実践的なインプリケーションを例示した。
       \end{itemize}
 \item 会計研究における PSM の利用 (第 3 節) の要約
       \begin{itemize}
        \item 観察不能なデータによって生じる内生性を緩和するための
              Heckman (1979) モデルの代わりとして、誤って利用しているケー
              スがしばしば確認される。
        \item デザインの選択を開示していない、あるいは、PSM と MR とでコ
              ントロール変数が不一致である研究が散見される。
       \end{itemize}
 \item デザインの選択 (第 4 節) の要約
       \begin{itemize}
        \item カットオフ・ポイント (cutoff point) が処置群を決定する場合、
              カットオフの近傍の観測値は over--represent し、第 II 種の
              過誤が生じる可能性が増大する。
        \item PSM による推定は変化しやすく (fickle) リプリケートが難しい
              ため、マッチド・サンプルに対して ``ストレス・テスト''
              (stress testing) を実施し、また、代替的なリサーチ・デザイ
              ンで PSM の追加検証を実施することが要求される。
       \end{itemize}
 \item 将来研究への提言
       \begin{itemize}
        \item 重要なデザインの選択を開示すること。
        \item PSM の利用目的を適切に理解すること。
        \item マッチド・サンプル以外の specification concerns も調査する
              こと。
        \item PSM に対して他のリサーチ・デザインをもとに追加分析を実施す
              ること、および、FFM に対処する新たな手法を模索すること。
       \end{itemize}
\end{itemize}