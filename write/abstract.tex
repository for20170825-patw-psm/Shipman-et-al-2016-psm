\begin{abstract}
会計研究において、
傾向スコア・マッチング (propensity score matching, PSM) は
平均処置効果 (ATEs) を推定するための一般的な手法として利用されるようになった。
本論文では、PSMの有用性と限界について、伝統的な重回帰 (MR) 分析と比較して議論する。
さまざまなPSMの設定 (design choice) について検討し、
2008〜2014年に主たる会計雑誌に掲載された、PSMを採用している
論文86本をレビューする。
2008年には、PSMを採用している論文数は0本であったのに対し、
2014年では26本と、顕著な増加が確認できる。
しかしながら、PSMの特性を過大評価したり、
重要な設定を開示していなかったり、および/あるいは、PSM の理論的背景を誤っ
 て解釈している研究が散見される。
そこで、はじめに、会計研究における3つの例から、
PSMの複雑性について実証的に明らかにする。
まず、処置群を表す変数 (treatment) がバイナリ変数でない場合、
PSMを利用することで、
効果量が最小になるようなサブサンプルを対象とした分析になってしまうという例を示す。
また、一見問題ないように思われる設定が、
サンプルの構成 (sample composition) および
ATEの推定に深刻な影響を及ぼすという例も提示する。
さいごに、
マッチング手法の利用を検討している将来の研究に対して、
いくつかの示唆を提供する。
\end{abstract}
